\clearpage
\phantomsection

\addcontentsline{toc}{chapter}{Tóm tắt}
\chapter*{\fontsize{13}{13}\selectfont{Tóm tắt}}
\fontsize{12}{12}\selectfont{
\noindent\textbf{Tóm tắt:}
Nghiên cứu này tập trung vào việc phân tích, phân loại và phát hiện các kiểu tấn công có khả năng xảy ra tại lớp ứng dụng của mạng Blockchain. Để tăng tính ứng dụng của nghiên cứu, nhóm lựa chọn hệ thống truy xuất nguồn gốc thực phẩm nông nghiệp trên nền tảng Blockchain để thực nghiệm và đánh giá các loại tấn công có thể xảy ra đối với một mạng Blockchain trong lĩnh vực nông nghiệp. Để tạo dữ liệu cho mô hình học, một mô hình mạng riêng tư Ethereum 2.0 đã được triển khai. Trong đó, dữ liệu của các giao dịch ở trạng thái thông thường được tạo bởi hệ thống truy xuất nguồn gốc mà nhóm đưa vào. Dữ liệu tấn công được tạo bởi các cuộc tấn công đã từng xảy ra gây thiệt hại lớn và được đánh giá là có khả năng gây hại đến hệ thống nông nghiệp trên nền tảng Blockchain. Với bộ dữ liệu thu thập được, nhóm đề xuất mô hình học máy sử dụng mạng neuron đa tầng với đầu vào rời rạc (SMLP) để có thể phân loại và phát hiện tấn công với phương pháp tiền xử lý dữ liệu áp dụng xử lý ngôn ngữ tự nhiên (NLP). Đồng thời, nhóm cũng sử dụng các mô hình phân loại cổ điển như SVM hay mô hình Ensemble để có thể tiến hành so sánh hiệu suất với phương pháp để xuất và kết quả thu được cho thấy phương pháp đề xuất cho ra hiệu suất tốt hơn so với các mô hình truyền thống, lên đến 99\% trong việc phân loại các tấn công. Kết quả này chỉ ra tính khả quan của mô hình cho việc áp dụng vào các mạng Blockchain thực tế

\vspace{0.5cm}
\noindent\textit{\textbf{Từ khóa:}} \textit{Bảo mật, Phát hiện tấn công, Blockchain, Hợp đồng thông minh, Giao dịch, Học máy.}
}